\documentclass{book} 

\usepackage{graphicx}
\usepackage{amsmath}
\usepackage{algorithm}
\usepackage{algpseudocode}
\usepackage{float}
\usepackage{hyperref}
\usepackage{placeins}
\usepackage{amssymb}
\usepackage{listings}
\usepackage{hyperref}
% For tables:
\usepackage{booktabs}

\usepackage{tikz}
\usetikzlibrary{arrows}
\usetikzlibrary{snakes}
\usetikzlibrary{decorations.pathmorphing}

\hypersetup{colorlinks=true,linkcolor=blue, linktocpage}

\title{Compound Interest}
\author{Saman Saadi}

\begin{document}
	\frontmatter
	\maketitle
	\mainmatter
	\section{Compound interst with only principal}
	Let's assume we want to invest with $p$ amount for principal. The anual interest rate is $r$. The anual interest rate is distributed $n$ times per year. We want to see how much our invevstment grows over $t$ years:
	
	\begin{equation*}
		p^{\prime} = p \times (1 + \frac{r}{n})^{n \times t}
	\end{equation*} 
	
	Let's explain the formula with an example. If $n = 12$ and $t=1$, then:
	
	\begin{equation*}
		\begin{split}
			p_0 &= p \\
			p_1 &= p_0 \times (1 + \frac{r}{n}) = p \times (1 + \frac{r}{n}) \\
			p_2 &= p_1 \times (1 + \frac{r}{n}) = p \times (1 + \frac{r}{n})^2 \\
			&\vdots \\
			p_{12} &= p_{11} \times (1 + \frac{r}{n}) = p \times (1 + \frac{r}{n})^{12}
		\end{split}
	\end{equation*}	 
	
	\section{Compound interest with principal and recurring contribution}
	We introduced a new variable. Let's assume $m$ is the number of times that we contribute $c$ in a year.
	\subsection{When $n = 1 \land m = 1$ }
	Let's assume $n=1$ and we have a recurring contribution of $c$ once every year ($m = 1$). Let's assume $e_i$ is $c$ as well as its growth if we start investing it in year $i$ for $t - i$ years:
	
	\begin{equation*}
		e_i = c \times (1 + r)^{t - i}
	\end{equation*}
	We assumed the contribution started at the end of the year. If it's at the beginning of the year, we call it $b_i$:
	
	\begin{equation*}
		b_i = c \times (1 + r)^{t - i + 1}
	\end{equation*}
	
	The total growth is:
	\begin{equation*}
		\begin{split}
			t_e &= p \times (1 + r)^t + \sum_{i = 1}^{t}{e_i} \\
			t_b &= p \times (1 + r)^t + \sum_{i = 1}^{t}{b_i}
		\end{split}
	\end{equation*}
	
	Let's calculate the second part of $t_e$:
	
	\begin{equation*}
		\begin{split}
			\sum_{i = 1}^{t}{e_i} &= \sum_{i = 1}^{t}{c \times (1 + r)^{t - i}} \\
			&= \sum_{i = 0}^{t - 1}{c \times (1 + r)^i} \\
			&= c \times \sum_{i = 0}^{t - 1}{(1 + r)^i} \\
			&= c \times (1 + (1 + r)^1 + (1 + r)^2 + \dots + (1 + r)^{t - 1}) \\
			&= c \times \frac{(1 + r)^t - 1}{(1 + r) - 1} \\
			&= c \times \frac{(1 + r)^t - 1}{r}
		\end{split}
	\end{equation*}
	
	Note that we use the fact that $\sum_{i = 0}^{t - 1}{(1 + r)^i}$ is \href[]{https://en.wikipedia.org/wiki/Geometric_series}{geometric series}. So we have:
	
	\begin{equation*}
		t_e = p \times (1 + r)^t + c \times \frac{(1 + r)^t - 1}{r}
	\end{equation*}
	
	With a similar approach we can calcualte $t_b$:
	\begin{equation*}
		\begin{split}
			\sum_{i = 1}^{t}{b_i} &= \sum_{i = 1}^{t}{c \times (1 + r)^{t - i + 1}} \\
			&= \sum_{i = 1}^{t}{c \times (1 + r)^i} \\
			&= c \times ((1 + r)^1 + (1 + r)^2 + \dots + (1 +r)^t) \\
			&= c \times (1 + r) \times (1 + (1 + r)^1 + (1 +r)^2 + \dots + (1 +r)^{t - 1}) \\
			&= c \times (1 + r) \times \frac{(1 +r)^t - 1}{r} \\
		\end{split}
	\end{equation*}
	
	So we have:
	
	\begin{equation*}
		t_b = p \times (1 + r)^t + c \times (1 + r) \times \frac{(1 +r)^t - 1}{r}
	\end{equation*}
	\subsection{General formula for recurring contribution}
	
	It's a bit difficult to calculate it very accurately. However, The following formula is reasonably accurate:
	
	\begin{equation*}
		\begin{split}
		t_e &= p \times (1 + \frac{r}{n})^{n \times t} + \sum_{i = 1}^{m \times t}{c \times (1 + \frac{r}{n})^{n \times t - i \times \frac{n}{m}}} \\
		t_b &= p \times (1 + \frac{r}{n})^{n \times t} + \sum_{i = 0}^{m \times t - 1}{c \times (1 + \frac{r}{n})^{n \times t - i \times \frac{n}{m}}}
		\end{split}
	\end{equation*}
\end{document}
